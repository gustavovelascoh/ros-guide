\chapter{Instalando ROS}

Las instrucciones que se presentan a continuación, son para instalar ROS Hydro Medusa en Ubuntu 12.04 (Precise Pangolin). Esta distribución de ROS solo es compatible con las distribuciones de Ubuntu 12.04 (Precise), 12.10 (Quantal) y 13.04 (Raring). 

\section{Configuración inicial de los repositorios}

Para empezar debemos configurar Ubuntu, específicamente la utilidad apt, para que reconozca los repositorios de ROS y nos permita instalarlo y actualizarlo. Por defecto en Ubuntu 12.04 (precise) ya están habilitados los repositorios universe, multiverse y restricted, lo cual era un paso adicional cuando se instalaba ROS en versiones anteriores de Ubuntu. Para verificarlo podemos abrir la interfaz gráfica de orígenes de software con el siguiente comando:

\begin{lstlisting}
$ python /usr/bin/software-properties-gtk
\end{lstlisting}

Aquí podemos verificar que están seleccionados los cuatro tipos de repositorios (main, universe, restricted y multiverse). Ahora, lo que debemos hacer es añadir los repositorios de ROS a la lista de fuentes de software y configurar el acceso con los siguientes comandos. 

\begin{lstlisting}
$ sudo sh -c 'echo "deb http://packages.ros.org/ros/ubuntu precise main" > /etc/apt/sources.list.d/ros-latest.list'
$ wget http://packages.ros.org/ros.key -O - | sudo apt-key add -
\end{lstlisting}

Ahora ya tenemos los repositorios de ROS configurados para su administración desde el comando apt.

\section{Instalación y configuración inicial}

Aunque ya configuramos los repositorios de ROS para su funcionamiento con apt, debemos corroborar que el listado de paquetes esté actualizado:

\begin{lstlisting}[language = bash]
$ sudo apt-get update
\end{lstlisting}

Debido a que ROS contiene gran cantidad de herramientas y utilidades, hay 4 opciones de instalación. La primera, la recomendada y la que vamos a usar, es la instalación completa (Desktop-full), la cual incluye todo el core de ROS, librerías genéricas de varios robots, simuladores 2D y 3D, herramientas de visualización y librerías de navegación y percepción.

\begin{lstlisting}[language = bash]
$ sudo apt-get install ros-hydro-desktop-full
\end{lstlisting}

La segunda opción (Desktop) incluye el core, las librerías de robots y las herramientas de visualización. 

\begin{lstlisting}[language = bash]
$ sudo apt-get install ros-hydro-desktop
\end{lstlisting}

La tercera opción (ROS-base) solamente tiene el core de ROS y librerías de comunicación, nada de interfaces gráficas (GUI). 

\begin{lstlisting}[language = bash]
$ sudo apt-get install ros-hydro-ros-base
\end{lstlisting}

La última opción es instalar paquetes individuales especificando el nombre en el comando:

\begin{lstlisting}[language = bash]
$ sudo apt-get install ros-hydro-PACKAGE
\end{lstlisting}

por ejemplo:

\begin{lstlisting}[language = bash]
$ sudo apt-get install ros-hydro-slam-gmapping
\end{lstlisting}

Para la  instalación de la versión completa se descargan aproximadamente 900 MB, que descargando a 1M/s se demoraría unos 15 minutos, y tras esto se configurarán los paquetes descargados, así que mientras se descarga, podemos ir por un café y leer un poco sobre la estructura y el funcionamiento de ROS si no lo hemos hecho :D

...

Después de haber finalizado la instalación completa, debemos configurar algunas herramientas adicionales y configurar nuestro entorno de trabajo.

Empezaremos inicializando rosdep, una herramienta que nos facilita la instalación de algunas dependencias de paquetes y que es necesaria para ejecutar algunos componentes de ROS

\begin{lstlisting}[language = bash]
$ sudo rosdep init
$ rosdep update
\end{lstlisting}

Es conveniente que cada vez que iniciemos sesión las variables de entorno de ROS sean cargadas automáticamente:

\begin{lstlisting}[language = bash]
$ echo "source /opt/ros/hydro/setup.bash" >> ~/.bashrc
$ source ~/.bashrc
\end{lstlisting}

Verificamos que ya están declaradas las variables de entorno de ROS:

\begin{lstlisting}[language = bash]
$ export | grep ROS
declare -x ROS_DISTRO="hydro"
declare -x ROS_ETC_DIR="/opt/ros/hydro/etc/ros"
declare -x ROS_MASTER_URI="http://localhost:11311"
declare -x ROS_PACKAGE_PATH="/opt/ros/hydro/share:/opt/ros/hydro/stacks"
declare -x ROS_ROOT="/opt/ros/hydro/share/ros"
\end{lstlisting}

Ahora instalaremos rosinstall, una herramienta que nos ayudará a descargar las fuentes de los paquetes y a instalarlos. Se descargarán aproximadamente 10 MB.

\begin{lstlisting}[language = bash]
$ sudo apt-get install python-rosinstall
\end{lstlisting}

Finalmente, tenemos ROS Hydro Medusa instalado y configurado en nuestro sistema Ubuntu Precise Pangolin.
