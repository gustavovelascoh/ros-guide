\chapter{Conceptos básicos en ROS}

Como ROS es un sistema tan grande y complejo, se puede dividir en tres niveles para su mejor comprensión y apropiación. El primero es el nivel del sistema de archivos el cual describe la forma en cómo ROS trabaja con los paquetes, su definición y  los archivos y carpetas que estos contienen. El segundo nivel es el nivel funcional, llamado grafo de computación, en el que se especifican los elementos que hacen parte de la ejecución, su relación y su arquitectura de comunicaciones. El tercer y último nivel es el nivel de la comunidad, lo que incluye el manejo del software en cuanto a la contribución de paquetes, las distribuciones, los repositorios, las listas de correo, entre otros.

\section{Sistema de archivos}
\subsection{Conceptos básicos}
---TODO---
\section{Grafo de computación}

El funcionamiento de ROS se basa en una arquitectura de comunicación peer-to-peer, entre unidades llamadas nodos,  los cuales se comunican entre sí intercambiando información. Para entender un poco más y profundizar en el funcionamiento de ROS es necesario describir algunos conceptos de este sistema, que son nodos (nodes), maestro (master), servidor de parámetros (parameter server), mensajes (messages), temas (topics) y servicios (services). A este conjunto de elementos junto a la arquitectura de comunicación se les llama grafo de computación  (Computation graph).

\subsection{Elementos en el grafo de computación}

Nodos:
Los nodos son los procesos que se ejecutan, estos se comunican entre sí usando los temas para la transmisión de los mensajes, también usan invocación de servicios y usando el servidor de parámetros. Los nodos están pensados para ser granulares, es decir, un sistema para controlar un robot suele posee muchos nodos; Un nodo puede controlar el sensor láser, otro nodo controla los motores, otro nodo puede realizar la localización, otro nodo realiza la planeación de rutas y otro nodo puede controlar la representación gráfica del sistema.

Maestro
El maestro ROS (ROS master) es el que provee al sistema de servicios de nombramiento  y registro a los nodos, además permite la búsqueda de los elementos en el grafo de computación. El maestro también realiza el seguimiento de los publicadores y suscriptores a los temas, así como de los servicios. El papel del maestro es permitir a los nodos encontrarse el uno al otro y una vez localizados, estos se pueden comunicar entre sí.

Servidor de Parámetros
El servidor de parámetros hace parte del maestro y se usa para almacenar datos en una ubicación central, accesible por los nodos en tiempo de ejecución. Usualmente es utilizado para almacenar valores estáticos y parámetros de configuración.

Mensajes
Los mensajes son datos que con los que se comunican los nodos entre sí. Estos mensajes son como estructuras de tipo C, que contienen campos tipados. Tambien los campos pueden ser arreglos de estos tipos y otros mensajes.

Temas
Los temas son los encargados de enrutar los mensajes.

--TODO--

\subsection{Funcionamiento}
---TODO---
\section{Comunidad}
---TODO---